\documentclass[10pt,a4paper]{article}
\usepackage[utf8]{inputenc}
\usepackage[english]{babel}
\usepackage{amsmath}
\usepackage{amsfonts}
\usepackage{amssymb}
\title{Planning algorithm}
\author{Ladislav Petera}
\date{\today}
\begin{document}
\maketitle

\section{Problem space}

The number of possible permutations of sub tasks within the sprints is
$$n = (n_{team} \cdot n_{sprint}) ^{\sum_{i=0}^{n_{task}}{w_i}} $$ where
\begin{itemize}
\item n - number of possible solutions
\item $n_{team}$ - number of teams
\item $n_{sprint}$ - number of sprints
\item $n_{task}$ - number of planned tasks
\item $w_i$ - work of given task
\end{itemize}

In our example we will get $n = (5 \cdot 11) ^{280} = 1.6 \cdot 10^{687}$.
For comparison - the estimated number of atoms in the known universe is around $10^{80}$

\paragraph{Let's consider a more realistic example:}

\begin{itemize}
\item 5.000 story points (50 projects 100 SP each)
\item 16 sprints (a year)
\item 6 teams
\end{itemize}

This will result in problem space $n = (6 \cdot 16) ^{5000} = 2.2 \cdot 10^{9911}$.
Looks like we are going to have to optimize a bit in order to be able to get some usable solution in reasonable time (smaller than few billion years).

\section{Planning constraints}

\paragraph{Hard constraints}

The hard constraints limit what is physically possible. The construction heuristics will initialize our solution within these constraints.

\begin{itemize}
\item team must have required skill
\item task dependency must not be violated
\item sprint velocity must not be exceeded
\item max task velocity per spring must not be exceeded
\item task cannot start before first possible sprint
\item one task can only be done by one team
\end{itemize}

\paragraph{Soft constraints}

The soft constraints create pressure towards the desirable output.

\begin{itemize}
\item minimize costs of delay
\item sprints should have high utilization
\end{itemize}

\section{Construction heuristic}

The construction heuristics phase quickly initializes our solution. We will pre sort the planning entities and then use the FIRST\_FIT\_DECREASING algorithm. This algorithm cycles through all the planning entities starting with the most difficult ones, initializing one entity at a time. It takes the best available value and continues with the next entity.

A meaningful sort of the entities matching the constraints highly improves the chances of this phase to produce a meaningful result:

\begin{itemize}
\item blocking tasks before blocked tasks
\item tasks of earlier projects before tasks of later projects
\item projects with higher delay costs before projects with lower delay costs
\end{itemize}

Splitting up tasks is smaller subtasks will make it easier to find a relatively good solution based solely on construction heuristic. However it will lead to problems during the meta heuristic.

\section{Metaheuristic}

Every metaheuristic will be facing the problem, that a move of an sub-task seldom leads to a score improvement because it is the last sub-task in a task which really matters. This will result in discarding moves which lead towards global minimum. Combination of this fact and the huge problem space leads to an almost impossible optimization task for the metaheuristic.

\subsection{Move selector}
Finding a good method for selecting the next move is crucial.

\paragraph{Brain storming:}
\begin{itemize}
\item priorize sub tasks of already selected task for move selection
\item priorize blockers of moved tasks for move selection
\end{itemize}

\subsection{Score function improvement}

Alternatively we need to make the sub tasks have an effect on the score.

\paragraph{Brain storming:}
\begin{itemize}
\item currently we calculate score sometimes based on the tasks, sometimes based on subtasks. Changing all to sub tasks might have a positive effect.
\end{itemize}


\end{document}