\documentclass[10pt,a4paper]{article}
\usepackage[utf8]{inputenc}
\usepackage[english]{babel}
\usepackage{amsmath}
\usepackage{amsfonts}
\usepackage{amssymb}
\title{Planning algorithm}
\author{Ladislav Petera}
\date{\today}
\begin{document}
\maketitle

\section{Problem space}

$$n = n_{team} \cdot n_{sprint} \cdot \sum_{i=0}^{n_{task}}{w_i} $$ where
\begin{itemize}
\item n - number of possible solutions
\item $n_{team}$ - number of teams
\item $n_{sprint}$ - number of sprints
\item $n_{task}$ - number of planned tasks
\item $w_i$ - work of given task
\end{itemize}

In our example we will get $n = 5 \cdot 11 \cdot 280 = 15.400$

\paragraph{Let's consider a more realistic example:}

\begin{itemize}
\item 50.000 story points (~40 person years)
\item 16 sprints (a year)
\item 6 teams
\end{itemize}

This will result in problem space $n = 6 \cdot 16 \cdot 50000 = 4.8 \cdot 10^6$.
This is a relatively small number considering the problem space size the planning algorithms normally deal with.

\end{document}